\documentclass[10pt, a4paper]{article}

% Math & symbols
\usepackage{amsmath,amssymb,amsfonts}
\usepackage{bm}          % Bold math

% Physics helpers
\usepackage{physics}     % Dirac bra-ket, derivatives, etc.

% Layout & graphics
\usepackage{geometry}    % Page layout
\usepackage{graphicx}    % Include figures

% Units & numbers
\usepackage{siunitx}     % Proper units and formatting

% Hyperlinks (load near the end)
\usepackage{hyperref}

\geometry{margin=1in}
\title{Analytic Slater-Type-Orbital Integrals and Hartree--Fock Derivations}
\author{Nicholas Wu \\[2pt]\small Department of Chemistry, University of California, Irvine}
\date{\today}


\begin{document}
\maketitle
\tableofcontents
\newpage


\section{Hamiltonian}


We consider both the one-electron and two-electron Hamiltonians in atomic units.

\subsection{One-Electron System (Hydrogen-like Atom)}
\[
\hat{H}^{(1)} = -\frac{1}{2} \nabla^2 - \frac{Z}{r}
\]
This consists of the kinetic energy of the electron and the Coulomb attraction to the nucleus.

\subsection{Two-Electron System (e.g., Helium Atom)}
\[
\hat{H}^{(2)} = -\frac{1}{2} \nabla_1^2 - \frac{1}{2} \nabla_2^2 - \frac{Z}{r_1} - \frac{Z}{r_2} + \frac{1}{|\mathbf{r}_1 - \mathbf{r}_2|}
\]
Where:
\begin{itemize}
  \item \( \nabla_i^2 \): Laplacian w.r.t. electron \( i \)
  \item \( r_i \): distance from nucleus to electron \( i \)
  \item \( \frac{1}{|\mathbf{r}_1 - \mathbf{r}_2|} \): electron-electron repulsion
\end{itemize}

\section{Matrix Elements in STO Basis}

Let basis functions be:
\[
\big| \ell, m; \zeta \big\rangle = r^{\ell} e^{-\zeta r} Y_{\ell m}(\theta, \phi).
\]
We define matrix elements for kinetic, potential, and overlap as:

\subsection{Kinetic Energy}
\[
\langle \ell_1, m_1; \zeta_1 | \hat{T} | \ell_2, m_2; \zeta_2 \rangle
= \delta_{\ell_1 \ell_2}\, \delta_{m_1 m_2}\; t(\ell_1, \zeta_1, \zeta_2).
\]
Let the radial part be \( \chi(r) = r^\ell e^{-\zeta r} \). In spherical coordinates,
\[
\nabla^2 \chi(r)
= \frac{1}{r^2} \frac{d}{dr} \!\left( r^2 \frac{d\chi(r)}{dr} \right) - \frac{\ell(\ell + 1)}{r^2}\, \chi(r).
\]
Using integration by parts one obtains
\[
t(\ell,\zeta_1,\zeta_2)
= -\frac{1}{2} \int_0^\infty \chi_1 \,(\nabla^2 \chi_2)\, r^2 dr
= \frac{1}{2} \int_0^\infty \big(\nabla \chi_1\big)\!\cdot\!\big(\nabla \chi_2\big)\, r^2 dr.
\]
With
\[
\frac{d\chi_i}{dr} = -\zeta_i e^{-\zeta_i r} r^\ell + \ell e^{-\zeta_i r} r^{\ell-1},
\]
the integral becomes
\[
t(\ell,\zeta_1,\zeta_2)
= \frac{1}{2}\!\int_0^\infty \! e^{-(\zeta_1+\zeta_2)r}
\Big(\zeta_1\zeta_2\, r^{2\ell+2}
- \ell(\zeta_1+\zeta_2) r^{2\ell+1}
+ \ell^2 r^{2\ell} - \ell(\ell+1) r^{2\ell}\Big) dr,
\]
and evaluates to
\[
t(\ell,\zeta_1,\zeta_2) =
\frac{1}{2}\!\left[
\zeta_1\zeta_2\,\frac{\Gamma(2\ell+3)}{(\zeta_1+\zeta_2)^{2\ell+3}}
- \ell(\zeta_1+\zeta_2)\,\frac{\Gamma(2\ell+2)}{(\zeta_1+\zeta_2)^{2\ell+2}}
+ \ell^2\,\frac{\Gamma(2\ell+1)}{(\zeta_1+\zeta_2)^{2\ell+1}}
- \ell(\ell-1)\,\frac{\Gamma(2\ell+1)}{(\zeta_1+\zeta_2)^{2\ell+1}}
\right]\!.
\]

\subsection{Potential Energy}
\[
\langle \ell_1, m_1; \zeta_1 | \hat{V} | \ell_2, m_2; \zeta_2 \rangle
= \delta_{\ell_1 \ell_2}\, \delta_{m_1 m_2}\; v(\ell, \zeta_1, \zeta_2),
\]
with
\[
v(\ell, \zeta_1, \zeta_2)
= -Z \int_0^\infty r^{2\ell + 1} e^{-(\zeta_1 + \zeta_2) r} dr
= -Z \,\frac{\Gamma(2\ell + 2)}{(\zeta_1 + \zeta_2)^{2\ell + 2}}.
\]

\subsection{Overlap}
\[
\langle \ell_1, m_1; \zeta_1 | \ell_2, m_2; \zeta_2 \rangle
= \delta_{\ell_1 \ell_2}\, \delta_{m_1 m_2}\; s(\ell, \zeta_1, \zeta_2),
\]
\[
s(\ell, \zeta_1, \zeta_2)
= \int_0^\infty r^{2\ell + 2} e^{-(\zeta_1 + \zeta_2) r} dr
= \frac{\Gamma(2\ell + 3)}{(\zeta_1 + \zeta_2)^{2\ell + 3}}.
\]

\subsection{Repulsion Integral}
\[
\big\langle \ell_1, m_1; \zeta_1;\, \ell_2, m_2; \zeta_2 \,\big|\, \ell_3, m_3; \zeta_3;\, \ell_4, m_4; \zeta_4 \big\rangle
\]

\subsubsection{Spherical (Angular) Part Using 3j Symbols}
Expand
\[
\frac{1}{|\mathbf{r}_1 - \mathbf{r}_2|} 
= \sum_{l=0}^{\infty} \sum_{m=-l}^{l} 
\frac{4\pi}{2l+1} 
\frac{r_<^l}{r_>^{l+1}} 
Y_{l m}^*(\Omega_1)\, Y_{l m}(\Omega_2).
\]
The angular factor is \footnote{The triple product of spherical harmonics is expressed in terms of Wigner 3j symbols}
\[
A = \sum_{l=0}^{\infty} \sum_{m=-l}^{l}
\sqrt{(2\ell_1+1)(2\ell_2+1)(2\ell_3+1)(2\ell_4+1)}
\begin{pmatrix}\ell_1 & l & \ell_2 \\ 0 & 0 & 0\end{pmatrix}
\begin{pmatrix}\ell_1 & l & \ell_2 \\ m_1 & -m & m_2\end{pmatrix}
\begin{pmatrix}\ell_3 & l & \ell_4 \\ 0 & 0 & 0\end{pmatrix}
\begin{pmatrix}\ell_3 & l & \ell_4 \\ m_3 & m & m_4\end{pmatrix}.
\]

\subsubsection{Radial Part of the Two-Electron Repulsion Integral}
Define \( a = \ell_1 + \ell_2 + 2 \), \( b = \ell_3 + \ell_4 + 2 \), \( p = \zeta_1 + \zeta_2 \), \( q = \zeta_3 + \zeta_4 \).
\[
I = \int_0^\infty\!\!\int_0^\infty 
\frac{r_<^l}{r_>^{l+1}}\, r_1^a r_2^b 
e^{-p r_1 - q r_2} \, dr_1 dr_2.
\]
Split into two regions:

\paragraph{Region I: \( r_1 < r_2 \)}
\[
I_1 =  \frac{1}{r_2^{l+1}}
\int_0^{r_2} r_1^{a + l} e^{-p r_1}\, dr_1
= \frac{1}{r_2^{l+1}}\,\frac{\gamma(a+l+1, p r_2)}{p^{a+l+1}},
\]
where \(\gamma\) is the lower incomplete gamma function.

\paragraph{Region II: \( r_2 < r_1 \)}
\[
I_2 = \int_{r_2}^{\infty} r_1^{a-l-1} e^{-p r_1}\, dr_1 \; r_2^l
= \frac{r_2^l}{p^{a-l}}\,\Gamma(a-l, p r_2),
\]
where \(\Gamma\) is the upper incomplete gamma function.\\

The radial part of the two-electron integral is
\[
R^{L}(ij|kl)
=\frac{(a-L-1)!}{p^{a-L}}\sum_{k=0}^{a-L-1}\frac{p^k}{k!}\frac{\Gamma(b+L+k+1)}{(p+q)^{b+L+k+1}} + \frac{(a+L)!}{p^{a+L+1}}\left(\frac{\Gamma(b-L)}{q^{b-L}} - \sum_{k=0}^{a+L}\frac{p^k}{k!}\frac{\Gamma(b-L+k)}{(p+q)^{b-L+k}}\right)
\]
where $\gamma$ and $\Gamma$ denote the lower and upper incomplete gamma functions, respectively.
\section{Verification with Simple Molecules}
\subsection{$1s$ orbital hydrogen}
For hydrogen in \(1s\) set \(\zeta_1 = \zeta_2 = 1\), \(Z = 1\), \(\ell = 0\):
\[
t = \tfrac{1}{2}\Big(\tfrac{\Gamma(3)}{2^3}\Big) = \tfrac{1}{2}\cdot \tfrac{2!}{8} = \tfrac{1}{8},\quad
v = -\tfrac{\Gamma(2)}{2^2} = -\tfrac{1}{4},\quad
s = \tfrac{\Gamma(3)}{2^3} = \tfrac{2}{8} = \tfrac{1}{4}.
\]
Divide by \(s\) to obtain expectation values:
\[
\langle T\rangle = \frac{1/8}{1/4} = \frac{1}{2},\qquad
\langle V\rangle = \frac{-1/4}{1/4} = -1,
\]
hence \(E = \frac{1}{2} - 1 = -\frac{1}{2} \approx -13\ \text{eV}\).

\subsection{Generalized eigenvalue problem for H using two STOs}
For two \(s\)-type STOs \(\chi_i(r)=e^{-\zeta_i r}\) (\(i=1,2\), \(Z=1\)):
\[
\mathbf{H}\,\vec{c} = E\,\mathbf{S}\,\vec{c},\qquad
\mathbf{S} = \begin{bmatrix} S_{11} & S_{12}\\ S_{21} & S_{22}\end{bmatrix},\quad
\mathbf{H} = \begin{bmatrix} H_{11} & H_{12}\\ H_{21} & H_{22}\end{bmatrix}
= \begin{bmatrix} T_{11}+V_{11} & T_{12}+V_{12}\\ T_{21}+V_{21} & T_{22}+V_{22}\end{bmatrix}.
\]
The secular equation \(\det(\mathbf{H}-E\mathbf{S})=0\) yields two energies (ground and first excited).

\subsection{Generalized Eigenvalue Problem for $n$ STOs}
For \(n\) \(s\)-type STOs (\(\ell=0\)), \(\chi_i(r)=e^{-\zeta_i r}\),
\[
S_{ij} = \int_0^\infty e^{-(\zeta_i+\zeta_j) r} r^2 dr = \frac{2}{(\zeta_i+\zeta_j)^3},
\quad
T_{ij} = \frac{\zeta_i \zeta_j}{(\zeta_i+\zeta_j)^3},
\]
\[
V_{ij} = -\int_0^\infty e^{-(\zeta_i+\zeta_j) r}\, r\, dr = -\,\frac{1}{(\zeta_i+\zeta_j)^2},
\qquad
H_{ij} = T_{ij}+V_{ij}.
\]
Solve \(H\vec{c}=E S \vec{c}\) to obtain \(n\) approximate energies and eigenvectors.

\section{Comparison: Hydrogenic \(\zeta = Z/n\) vs. Even-Tempered Basis}
For hydrogen (\(Z=1\)), the exact \(1s\) is \(\chi_{1s}(r)=e^{-Zr}\) so \(\zeta=1\).
\subsection{Hydrogenic Orbital Energy (Exact \(\zeta = Z\))}
\(\langle T \rangle = \tfrac{1}{2},\; \langle V \rangle = -1,\; E=-\tfrac{1}{2}\).
\subsection{Even-Tempered STO Basis}
Choose \(\zeta_i=\alpha \beta^{i-1}\) (\(i=1,\dots,n\)). Typical \(\alpha=0.3\), \(\beta=2.0\) \(\Rightarrow\)
\(\zeta = [0.3,0.6,1.2,2.4,\dots]\).
\subsection{Numerical Comparison via Generalized Eigenvalue Problem}
Using the above \(S_{ij}\), \(T_{ij}\), and \(V_{ij}\),
solve \(H\vec{c}=ES\vec{c}\); the lowest eigenvalue approaches \(-0.5\) Hartree as \(n\) increases.

\section{Hartree--Fock for a Two-Electron System in a Non-Orthonormal Basis}
For a singlet Slater determinant
\[
\Psi^{\text{HF}}(x_1,x_2) = \varphi(\mathbf r_1)\,\varphi(\mathbf r_2)\,\Xi(\sigma_1,\sigma_2),\qquad
|\varphi\rangle = \sum_\mu C_\mu |\chi_\mu\rangle,
\]
we enforce
\[
\langle \varphi|\varphi\rangle = \sum_{\mu\nu} C_\mu^* C_\nu S_{\mu\nu} = 1,\qquad
S_{\mu\nu}=\langle \chi_\mu|\chi_\nu\rangle.
\]

\subsection{Hartree--Fock Lagrangian}
\[
E^{\text{HF}}[C] = 2\langle \varphi|\hat h|\varphi\rangle
+ \langle \varphi\varphi|\hat v_{ee}|\varphi\varphi\rangle
- \langle \varphi\varphi|\hat v_{ee}|\varphi\varphi\rangle_{\text{ex}},
\quad
\mathcal{L}[C,C^*] = E^{\text{HF}}[C] - \lambda\big(\langle \varphi|\varphi\rangle -1\big).
\]
In a basis,
\[
\mathcal{L}[C,C^*] = 
2\sum_{\mu\nu} C_\mu^* C_\nu\, h_{\mu\nu}
+ \sum_{\mu\nu\lambda\sigma} C_\mu^* C_\nu^* C_\lambda C_\sigma
\Big[ (\mu\nu|\lambda\sigma) - (\mu\nu|\sigma\lambda) \Big]
- \lambda\Big(\sum_{\mu\nu} C_\mu^* C_\nu\, S_{\mu\nu}-1\Big),
\]
where $h_{\mu\nu}$ are one-electron integrals (overlap, kinetic, nuclear attraction) and
$(\mu\nu|\lambda\sigma)$ are two-electron repulsion integrals.\footnote{In practice, the one-electron integrals admit closed forms; the two-electron terms combine angular-momentum algebra with radial integrals that may use incomplete gamma functions and numerical quadrature.}

\subsection{Stationary Condition}
\[
\sum_\nu \!\left(
h_{\mu\nu}
+ \sum_{\lambda\sigma} D_{\lambda\sigma}
\big[ (\mu\nu|\lambda\sigma) - \tfrac12 (\mu\lambda|\nu\sigma) \big]
\right) C_\nu
= \lambda \sum_\nu S_{\mu\nu} C_\nu,
\]
with spin-summed density
\[
D_{\lambda\sigma} = \sum_p n_p\, C_{\lambda p} C_{\sigma p}^*
\]
$n_p$ \text{ need not be integers (fractional occupations may be used to preserve symmetry).}\footnote{For example, in Boron the three degenerate $2p$ orbitals can be assigned occupations of $1/3$ each to maintain spherical averaging.}

\subsection{Roothaan--Hall Equations}
\[
F_{\mu\nu} = h_{\mu\nu} + \sum_{\lambda\sigma} D_{\lambda\sigma}
\Big[ (\mu\nu|\lambda\sigma) - \tfrac12(\mu\lambda|\nu\sigma) \Big],
\qquad
\mathbf F \mathbf C = \mathbf S \mathbf C \boldsymbol\varepsilon,
\]
solved self-consistently.\footnote{At every iteration the Coulomb and exchange matrices, and hence the Fock matrix, are symmetrized to ensure Hermiticity.}

\section{SCF Loop Implementation Steps}
\begin{enumerate}
    \item \textbf{System setup.}  
    Choose exponents $\zeta_i$ and angular labels $(\ell_i,m_i)$, nuclear charge $Z$, and orbital occupations. For each $\ell$, include all $m$. Even-tempered exponent sequences may be used and optimized variationally.\footnote{Optimization can be carried out via finite-difference Newton--Raphson on the even-tempered parameters to lower the total energy.}

    \item \textbf{One- and two-electron quantities.}  
    Build $S$ and $H_{\text{core}}=T+V$; precompute the electron repulsion integrals (ERI) once before the SCF loop.\footnote{Precomputation is costly but avoids recomputation inside the loop and improves stability.}

    \item \textbf{Initial guess.}  
    Start from the symmetrized core Hamiltonian as the initial Fock matrix. Construct the density blockwise in the $s$ and $p$ subspaces according to the specified occupations, then embed and sum to obtain the full (block-diagonal) density.\footnote{Blockwise construction helps maintain the intended $s/p$ symmetry and simplifies the initial diagonalizations.}

    \item \textbf{SCF iterations.}  
    Repeat until convergence:
    \begin{enumerate}
        \item Build Coulomb $J$ and exchange $K$ from the current density and symmetrize.
        \item Form \(F = H_\text{core} + J - \tfrac{1}{2}K\) and symmetrize.
        \item Solve the generalized eigenvalue problems (blockwise if applicable) and update the density from the occupied orbitals (including any fractional occupations).
        \item Compute
        \(E_\text{one} = \mathrm{Tr}[D H_\text{core}]\),\;
        \(E_\text{two} = \tfrac12 \mathrm{Tr}[D(J - \tfrac12 K)]\),\;
        \(E_\text{total} = E_\text{one} + E_\text{two}\).
        \item Check convergence: require both the energy change \(\Delta E\) and density change \(\Delta D\) to be below thresholds; otherwise continue with the updated density.
    \end{enumerate}

    \item \textbf{Reporting.}  
    On convergence, report \(E_\text{one}\), \(E_\text{two}\), and \(E_\text{total}\). If the maximum number of iterations is reached without meeting thresholds, return the last values with a warning.\footnote{Iteration logs display the energy and residual changes, and indicate whether convergence was achieved.}
\end{enumerate}

\section{Newton--Raphson Optimization of Even--Tempered Exponents}
\subsection{1. Even--tempered exponents}
Define
\[
\zeta_i(\alpha_0, \beta) = \alpha_0 \beta^{\,i-1}, \quad i=1,\dots,n,
\qquad \alpha_0>0,\;\beta>1.
\]
\subsection{2. Objective function}
Let \(E(\{\zeta_i\})\) be the SCF total energy; define
\[
f(\alpha_0,\beta)=E\!\big(\{\zeta_i(\alpha_0,\beta)\}_{i=1}^n\big),\qquad
\min_{\alpha_0>0,\;\beta>1} f(\alpha_0,\beta).
\]
\subsection{3. Finite difference derivatives}
With steps \(h_\alpha,h_\beta\),
\begin{align}
\frac{\partial f}{\partial \alpha_0} &\approx 
\frac{f(\alpha_0 + h_\alpha, \beta) - f(\alpha_0 - h_\alpha, \beta)}{2 h_\alpha}, \\
\frac{\partial f}{\partial \beta} &\approx
\frac{f(\alpha_0, \beta + h_\beta) - f(\alpha_0, \beta - h_\beta)}{2 h_\beta}, \\[4pt]
\frac{\partial^2 f}{\partial \alpha_0^2} &\approx 
\frac{f(\alpha_0 + h_\alpha, \beta) - 2 f(\alpha_0, \beta) + f(\alpha_0 - h_\alpha, \beta)}{h_\alpha^2}, \\
\frac{\partial^2 f}{\partial \beta^2} &\approx 
\frac{f(\alpha_0, \beta + h_\beta) - 2 f(\alpha_0, \beta) + f(\alpha_0, \beta - h_\beta)}{h_\beta^2}, \\
\frac{\partial^2 f}{\partial \alpha_0 \, \partial \beta} 
&\approx 
\frac{ f(\alpha_0 + h_\alpha, \beta + h_\beta) 
- f(\alpha_0 + h_\alpha, \beta - h_\beta) 
- f(\alpha_0 - h_\alpha, \beta + h_\beta) 
+ f(\alpha_0 - h_\alpha, \beta - h_\beta) }{4 h_\alpha h_\beta}.
\end{align}
By Schwarz’s theorem, the mixed derivative is symmetric.

\subsection{4. Newton--Raphson update}
Let
\[
\bm{x}=\begin{bmatrix}\alpha_0\\ \beta\end{bmatrix},\quad
\nabla f(\bm{x})=\begin{bmatrix}\partial f/\partial \alpha_0\\ \partial f/\partial \beta\end{bmatrix},\quad
H(\bm{x})=\begin{bmatrix}
\partial^2 f/\partial \alpha_0^2 & \partial^2 f/\partial \alpha_0 \partial \beta \\
\partial^2 f/\partial \beta \partial \alpha_0 & \partial^2 f/\partial \beta^2
\end{bmatrix}.
\]
The update is
\[
\Delta \bm{x} = - H(\bm{x})^{-1}\, \nabla f(\bm{x}),\qquad
\bm{x}_{k+1} = \bm{x}_k + \Delta \bm{x}.
\]

\subsection{5. Constraints and stability}
\begin{itemize}
    \item If \(\alpha_0 \le 0\) or \(\beta \le 1\), reject the step or apply backtracking.
    \item Optionally regularize \(H\) with a small ridge term \(H_\text{reg}=H+\lambda I\) for numerical stability.
    \item Stop when \(\|\nabla f(\bm{x})\|<\varepsilon\), or \(|\Delta \bm{x}|<\delta\), or \(|f_{k+1}-f_k|<\eta\).
\end{itemize}

\end{document}

\documentclass[10pt, a4paper]{article}

% ====================
% Packages
% ====================
\usepackage[margin=1in]{geometry}
\usepackage{graphicx}   % for including plots
\usepackage{amsmath,amssymb}
\usepackage{booktabs}   % for professional tables
\usepackage{siunitx}    % for scientific units
\usepackage{hyperref}   % for clickable refs/links
\usepackage{caption}
\usepackage{subcaption}

% ====================
% Title Info
% ====================
\title{Hartree--Fock with Slater-Type Orbitals: Results and Discussion}
\author{Your Name}
\date{\today}

% ====================
% Document
% ====================
\begin{document}
\maketitle

\begin{abstract}
This document presents the computational results obtained from the
implementation of Hartree--Fock theory using Slater-type orbitals
(STOs). Energies for hydrogen, helium, and boron are reported,
with convergence behavior analyzed and compared to benchmark values.
\end{abstract}

\tableofcontents
\newpage

% ====================
\section{Introduction}
Briefly describe the purpose of this document:
to summarize the numerical results of the Hartree--Fock
implementation (details of derivation are in a separate document).
State goals:
\begin{itemize}
    \item Test implementation on hydrogen, helium, and boron.
    \item Compare with reference Hartree--Fock values.
    \item Analyze basis set dependence and convergence.
\end{itemize}

% ====================
\section{Computational Setup}
\begin{itemize}
    \item Systems studied: H, He, B.
    \item Basis functions: even-tempered STOs with parameters $(\alpha, \beta)$.
    \item Occupations: note fractional occupations for boron $2p$ orbitals.
    \item Convergence threshold: $10^{-6}$ Hartree in energy and density.
    \item Maximum SCF iterations: 50.
\end{itemize}

% ====================
\section{SCF Convergence Behavior}
\subsection{Iteration Logs}
Include representative SCF iteration outputs for each system.  
For example, plot energy vs. iteration:
\begin{figure}[h!]
    \centering
    %\includegraphics[width=0.7\textwidth]{boron_scf_convergence.pdf}
    \caption{SCF energy convergence for boron with 10 $s$- and 10 $p$-type basis functions.}
    \label{fig:scf_boron}
\end{figure}

\subsection{Convergence Discussion}
Comment on whether convergence was smooth, oscillatory, or required many iterations.

% ====================
\section{Energies and Comparisons}
\subsection{Hydrogen and Helium}
Insert table comparing computed vs. exact HF energies:
\begin{table}[h!]
    \centering
    \sisetup{round-mode=places,round-precision=6}
    \begin{tabular}{lcccc}
        \toprule
        Atom & Basis Size & $E_\text{SCF}$ (Ha) & Ref. HF (Ha) & \% Error \\
        \midrule
        H    & 5 STOs   & -0.4998 & -0.5000 & 0.04\% \\
        He   & 5 STOs   & -2.8600 & -2.8617 & 0.06\% \\
        \bottomrule
    \end{tabular}
    \caption{Comparison of computed SCF energies with benchmark values.}
    \label{tab:H_He}
\end{table}

\subsection{Boron}
Report final SCF energy for boron, compare with literature.  
Discuss role of fractional occupation of $2p$ orbitals.

% ====================
\section{Discussion}
\begin{itemize}
    \item Accuracy of results relative to expectations.
    \item Basis set dependence (energy lowers with more STOs).
    \item Challenges (e.g., convergence without DIIS, cost of ERIs).
    \item Physical interpretation (fractional occupations maintain spherical symmetry).
\end{itemize}

% ====================
\section{Conclusion}
Summarize:
\begin{itemize}
    \item SCF implementation successfully reproduces known HF energies.
    \item Boron calculations highlight need for fractional occupation.
    \item Future improvements: DIIS acceleration, Gaussian basis comparison, molecular extension.
\end{itemize}

% ====================
\appendix
\section{Raw SCF Logs}
Include snippets of iteration outputs for completeness.

\section{Additional Plots}
Add convergence plots or orbital eigenvalues here.

\end{document}
